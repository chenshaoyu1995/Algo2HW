\begin{problem}{4} ~\\
Since INSERT and DELETE have running time that is asymptotically the same, we can assume the running time is t. Starting with an empty data structure, every time an INSERT is carried out, we charge it 2t instead of t, to ensure the potential DELETE operation later when the size before DELETE is the same as that after INSERT can be free. In the worst case, there is nothing left in the data structure at the end and all deposit is used. If the data structure is not empty at the end, it implies there is some deposit left. In other words, it is impossible to overdraw the account.\\
\\
However, the same trick does not work in reverse. The first operation on an empty data structure can only be INSERT. If its amortized cost is free, that means the INSERT is using some deposit of the data structure. An empty data structure has no deposit and deposit cannot be overdrawn. Therefore, it is unreasonable to make INSERT operation free.
\end{problem}